\documentclass[output=collectionpaper,hidelinks]{langscibook}

\normalem
\theoremstyle{remark}

\hyphenation{Nilo-Saharan nom-i-nal Eth-iopia Ko-man evi-nces Su-dan gen-der none-the-less morph-ology morph-olo-gi-cal morph-olo-gi-cally mar-kers case-marking cross-reference se-man-tic assign-ment predict-able apell-atives demon-strative pro-noun de-fault num-ber borr-owed index-ation Erg-ative trans-itive rel-a-tive temp-oral adverb-ial sub-ordinate in-trans-itive pro-nouns alter-na-tions clitic-i-zation gram-ma-ti-ca-lized None-the-less occasion-al care-ful}

\newtheorem*{principle-distinct}{Principle of Fewer Distinctions}
\newtheorem*{principle-transparency}{Principle of One-Meaning–One-Form}
\newtheorem*{principle-independence}{Principle of Independence}


\title{Gender in Uduk}
\author{%
 Don Killian  \affiliation{University of Helsinki}
}

\begin{document}

\abstract{%
\label{firstpage:Killian}
Uduk, a Koman language spoken on the border of Ethiopia and Sudan, evinces a
number of unusual characteristics in its system of gender marking.  Uduk has two
gender classes, with agreement displayed primarily in the verbal system and
adjacent case-marking particles.  In contrast to related Koman languages,
however, semantics play a minimal role in class assignment, unrelated to
biological sex.  Furthermore, as biological sex does not play a role in gender
assignment in general, personal pronouns do not differentiate gender in any
person.  Instead, all personal pronouns are assigned to Class 1 in the same
manner that nouns would be.  Lastly, Uduk shows some unorthodox aspects in the
way it indexes gender on verbs, using what might be considered subtractive
morphology.

This article looks at the complexity and features of gender in Uduk from a
typological perspective; despite some unorthodox and atypical typological
features, however, the system does not appear to be complex.
\medskip

\textbf{Keywords:}
Uduk, gender, assignment, Koman, adjacency, ditropic.
}%
\maketitle


\section{Background}
\label{sec:Background}

Koman languages form a small language family spoken along the borderland area of
Ethiopia, Sudan and South Sudan.  The family is comprised of four living
languages: Gwama (Kwama) [kmq], Tʼapo (also known as Opo or Opuo) [lgn], Komo
[xom] and Uduk (Tw’ampa), [udu].  A fifth language which is now extinct, Gule,
was placed into Koman by Greenberg with relatively little data available
\citep{hv:Greenberg:Africa:1963}, and its placement in Koman is tentative.

The presence of gender distinctions on pronouns in Koman languages was noted
early on, but no research until recently has uncovered any signs of a nominal
grammatical gender system, which all extant Koman languages have in some
fashion.%
\footnote{The Yabus dialect of Uduk appears to be an exception to this, and does
not have any grammatical gender.} %
The data on Uduk presented here is based on thirteen months of fieldwork between
2011 and 2014 in Ethiopia.

\section{Introduction}
\label{sec:Introduction}


Gender is a noun classification strategy in which nouns are encoded to belong to
a particular lexical class, which is further ``reflected in the behavior of
associated words.'' \citep[231]{Hockett1958}. This is commonly referred to as
\emph{agreement}, a relationship in which one element takes an inflectional form
determined by semantic or morphosyntactic properties of another element.
Following \cite{Corbett2006}, the element which determines the agreement is the
\emph{controller}, and the element whose form is determined by agreement is the
\emph{target}.

As the notion of agreement implies that the controller is present (cf.\
\citealt{Corbett2006}), the term \emph{indexation} is used instead of agreement.
Indexation is defined here as the morphosyntactic realization of a controller's
capacity to control a target, with the controller being either present or
recoverable or identifiable in some way.  This may be done inflectionally
through means of an affix or clitic, but this may also occur on a broader level
by use of particular constructions, as Uduk does not always index gender on
targets through inflectional markers.  In particular, when in object position,
one class of nouns actually constrains verb paradigms, limiting the possible
\emph{subject} cross-referencing markers on the verb. Thus, it is possible to determine the gender of the object from the morphology of the verb, despite there being no affix on the verb expressing gender agreement with the object.

Many other aspects of the Uduk gender system show themselves to be unorthodox in
nature. Semantic assignment exists only for a very small part of the lexicon,
formal assignment (in terms of word formation rules) for another very small
part, with the rest being largely arbitrary. Semantics in general play a smaller
role than usual in gender assignment, and Uduk's cut-off point in the animacy
hierarchy for semantic assignment is higher than simply `human'.

Furthermore, typical typological indexation targets of gender include
demonstratives, determiners, personal pronouns, relative pronouns, adjectives,
verbs \citep{DiGarbo2014}. For Uduk, the only target in this list is verbs. In addition to verbs, indexation is primarily indicated on a single clitic or particle which immediately precedes the controller, and on prepositions

It is worth considering Uduk's gender system in terms of its linguistic
complexity.%
\footnote{Linguistic complexity refers here to the amount of
information needed to describe the system, following e.g.\ \citet{Dahl2004} and \citet{Miestamo2008}.} %
Following \citet[50]{DiGarbo2016}, I look at three
general principles governing local complexity:

\begin{principle-distinct}
\hfill \break
Everything else being equal, a grammatical domain with \emph{n} distinctions is less complex than one with \emph{n+1} distinctions.
\end{principle-distinct}

\begin{principle-transparency}
\hfill \break
Everything else being equal, a grammatical entity with \emph{n} forms is less complex than one with \emph{n+1} forms. \newline
Everything else being equal, a grammatical entity with \emph{n} meanings is less complex than one with \emph{n+1} meanings.
\end{principle-transparency}

\begin{principle-independence}
\hfill \break
Everything else being equal, a grammatical domain that is independent of semantic and functional properties of other domains is less complex than a grammatical domain that is dependent on \emph{n} or \emph{n + 1} semantic and functional properties of other grammatical domains.
\end{principle-independence}

In addition to the metrics proposed above, there are at least two additional
factors which may play a role, arbitrariness and adjacency, although how they
fit precisely remains to be determined. All of these are discussed further in
\sectref{sec:Complexity}.

\section{Introduction to gender in Uduk}
\label{sec:Introduction_to_gender_in_Uduk}

All nouns in Uduk, including proper nouns, are allocated to one of two
possible grammatical gender classes, labeled \emph{Class 1} and \emph{Class
2}.

Gender in Uduk is covert, and not marked directly on nouns. Gender distinctions
are seen most commonly through the presence or absence of the Class 2 clitic
\emph{à=};%
\footnote{Transcriptions used here follow the IPA, except for <y>, which
represents IPA \emph{j}, and <j> which represents IPA \emph{ɟ}.} %
this marker is, however, is optional in isolation.  Furthermore, if gender is
indexed on a previous word in the phrase, then \emph{à=} is not used with the
noun.  Vocative use also neutralizes gender distinctions in many instances.
When directly addressing an individual, all personal names\footnote{All personal
names are assigned to Class 2, discussed in more detail in
\sectref{sec:Assignment}.} and most Class 2 kinship terms remove \emph{à=}; a
handful of kinship terms may retain \emph{à=} to indicate a type of
informality.  In all other known instances than these, \emph{à=} occurs with
Class 2 nouns.

Gender indexation primarily occurs on case-marking clitics or particles which
immediately precede the controller.  Prepositions, conjunctions, and
complementizers also undergo a simple phonological alternation, depending on the
gender of the noun that follows, and verbs also vary in their conjugation
paradigms depending on the gender of a postverbal object.  In some instances,
clitics may be considered ditropic clitics,%
\footnote{Ditropic clitics are a type of clitic which occur before a particular
lexical class or syntactic phrase functionally related to the clitic in
question, but the clitics nonetheless phonologically attach to the constituent
on the `other' side instead.  This host generally is structurally and
functionally highly variable, and shows little functional relation to the
clitic.  For more details, see \citet{Cysouw_Enclitics_2005}.} %
phonologically attaching to the constituent which immediately precedes the
clitic.  However, unlike more typical situations of ditropic clitics,
phonological hosts are more constrained in Uduk.  Further details are discussed
in \sectref{sec:Don:Case_Marking} below after a general introduction to
grammatical relations in Uduk.


\subsection{Grammatical relations overview}
\label{sec:Don:Grammatical_relations_overview}

Case and constituent order are intertwined in Uduk, and it is not possible to
discuss one without the other. The order of constituents frequently changes,
and the order of the arguments affects the way in which these are
encoded.%
\footnote{The framework used here to refer to argument structure is based
on a division elaborated on by \citet{Dixon_1994}, in which participants of a
clause are divided into core and peripheral roles. Core functions include the
transitive subject (A), the intransitive subject (S), and the transitive object
(O); all other participants are treated as peripheral.}

Uduk follows a verb-second pattern similar to that of some neighboring Nilotic
languages. Intransitive clauses primarily use SV order, with occasional
instances of VS order in specific types of subordinate clauses. Transitive
clauses regularly alternate between OVA and AVO, and cannot be easily
characterized as having a dominant constituent order. Other constituent orders
do not occur in main clauses.

The only position in a clause in which a noun receives morphological case is the
argument immediately following a verb. Other core relations are not
case-marked, irrespective of whether they occur before or after the immediately
postverbal position. If the postverbal argument is O, this may be indicated
by an Accusative ditropic clitic which phonologically attaches onto the verb.
If the argument is A, the verb is marked by a ditropic clitic indicating Ergative
case.%
\footnote{The Ergative case primarily indicates the subject of a transitive
clause; however, in two instances, namely relative clauses and temporal
adverbial subordinate clauses, the same marker is also used with subjects of
intransitive clauses as well. In these two clause types, then, Uduk would be
considered as having Marked Nominative case marking rather than Ergative. All
Marked Nominative examples are nonetheless glossed as \textsc{erg}, however, to
simplify matters. For further details, see \citet{Killian2015}.} %
Note
that verbs ending in vowels add a nasal suffix if the argument that follows is
marked with Ergative case.

Table \ref{tab:Don:case_markers} shows the different case markers used in
Uduk.%
\footnote{Absolutive is not used here to refer to a case encompassing S and O,
but is used in a more general sense to refer to most situations in which the
noun is not marked for Accusative, Associative, Ergative, or Genitive.  This
includes all preverbal arguments and second arguments after the verb in
ditransitive constructions.  Absolutive Class 2 \emph{à=} is not used in
prepositional phrases, however, and optionally in citation form.  Associative is
used to refer to a type of noun-noun collocations in which the second noun
modifies the first in some way, typically conveying either possession or
association.  It is similar to the Genitive, but the relationship between the
two nouns in the Associative is much broader and less defined.  For further
details, see \citet{Killian2015}.} %
All case-marking enclitics are ditropic.

\begin{table}[h!]
\centering
\caption{Case Markers}
\label{tab:Don:case_markers}
\begin{tabularx}{.7\textwidth}{lXXXXX}
\lsptoprule
& (Abs.) & Acc. & Ass. & Erg. & Gen. \\
\midrule
Class 1 & ø & ø & ø & \itshape  =ā & \itshape  gì \\
Class 2 &  \itshape à= & \itshape  =ā &  \itshape =ā &  \itshape =mā & \itshape  =mā \\
\lspbottomrule
\end{tabularx}
\end{table}

Some examples are as follows:%
\footnote{The underlined argument indicates the
topical argument of a transitive clause.}

\ea
Intransitive \\
\gll à=cʼí kʼūtʰ-úɗ \\
 \textsc{cl2}=child(\textsc{cl2}) cough:\textsc{ipfv}-\textsc{3sg} \\
\glt `The child coughed.' \\
\z

\ea
 Transitive, AVO order \\
 \gll à=ɲáw ūr-úɗ=ā tʼíkʰ \\
 \textsc{cl2}=cat(\textsc{cl2}) chase:\textsc{ipfv}-\textsc{3sg}=\textsc{acc.cl2} rat(\textsc{cl2}) \\
\glt `The \uline{cat} chased the rat.' \\
\z

\protectedex{%
\ea
 Transitive, OVA order \\
\gll à=ɲáw wǔcʼ=mā kʼá \\
 \textsc{cl2}=cat(\textsc{cl2}) bite:\textsc{pfv}=\textsc{erg.cl2} dog(\textsc{cl2}) \\
\glt `The dog bit the \uline{cat}.' \\
\z
}%


\subsection{Gender and case marking}
\label{sec:Don:Case_Marking}

As mentioned in the previous section, gender differentiations are found in case
marking. Uduk encodes gender and case marking cumulatively, with a single
combined morph to represent multiple features. Case is generally marked by
clitics or particles immediately preceding the noun, and case markers which
indicate core arguments only occur in the immediately postverbal position.

All case markers except Class 2 Absolutive \emph{à=} and Class 1 Genitive
\emph{gì} are ditropic clitics, clitics which form phonological units with the
immediately preceding element. Not all markers, however, are as bound as
others, and boundedness forms something of a continuum.

Accusative Class 2 \emph{ā} and Ergative Class 1 \emph{ā} both form relatively
tight-knit phonological units with the verb, and trigger morphophonological
changes on the verb.%
\footnote{Glottalized consonants in word-final position are
unreleased. If any affixes or clitics are placed after them, they undergo a
morphophonological alternation described in more detail in
\citet[48]{Killian2015}.} %
If a verb ends in a vowel, however, Accusative \emph{ā} does behave slightly
differently compared to the Ergative \emph{ā}.  Verbs ending in a vowel always
add an extra \emph{-n} to the end when occurring before Ergative case markers of
either class, before Class 1 \emph{ā} as well as before Class 2 \emph{mā}.
Accusative Class 2 \emph{ā} on the other hand simply attaches to whatever the
final consonant or vowel is, including other vowels.  Associative Class 2
\emph{ā} behaves identically to Accusative phonologically, but attaches to a
noun rather than a verb.%
\footnote{If the first noun in the Associative construction ends in a vowel and
the consonant of the second noun begins with a plosive, a homorganic nasal is
used in place of \emph{ā}.  For more details, see \citet[89]{Killian2015}.}

All case markers discussed except for Genitive Class 1 \emph{gì} undergo
phonological tonal alternations depending on the immediately preceding tone.
This includes Accusative Class 2 \emph{ā}, Associative Class 2 \emph{ā},
Ergative Class 1 \emph{ā}, Accusative Class 2 \emph{mā}, and Genitive Class 2
\emph{mā}.  The base tone of the case marker is mid, but lowers to low when
immediately following a low tone.  Neither Ergative Class 2 \emph{mā} nor
Genitive Class 2 \emph{mā} trigger morphophonological changes, however.

Genitive Class 1 \emph{gì} is not a clitic, but rather an independent particle
which does not change tone or affect any consonants or tones around it.

Some simple examples of each form are given below.%
\footnote{Clauses with Class 1
postverbal objects are not included, as they are a special case discussed in
 \sectref{sec:Don:Verbs} below.}

\ea
 Accusative, Class 2 \\
\gll kʼwāní lǒɓ-ón=\textbf{ā} kʰúrā \\
 people(\textsc{cl1}) play:\textsc{ipfv}-\textsc{3pl}=\textbf{\textsc{acc.cl2}} ball(\textsc{cl2}) \\
\glt `The \uline{people} are playing football.' \\
\z

\ea
 Ergative, Class 1 \\
\gll à=kʰúrā lǒɓ=\textbf{ā} kʼwāní \\
 \textsc{cl2}=ball(\textsc{cl2}) play:\textsc{ipfv}=\textbf{\textsc{erg.cl1}} people(\textsc{cl1}) \\
\glt `The people are playing \uline{(foot)ball}.' \\
\z

\ea
 Ergative Class 2 \\
\gll à=kʰúrā lǒɓ=\textbf{mā} cʼí \\
 \textsc{cl2}=ball(\textsc{cl2}) play:\textsc{ipfv}=\textbf{\textsc{erg.cl2}} child(\textsc{cl2}) \\
\glt `The child is playing \uline{(foot)ball}.' \\
\z

\ea
 Genitive, Class 1 \\
\gll à=nós \textbf{gì} wàt̪íʔ \\
 \textsc{cl2}=pot(\textsc{cl2}) \textbf{\textsc{gen.cl1}} man(\textsc{cl1}) \\
\glt `The man's pot.' \\
\z

\ea
 Genitive, Class 2 \\
\gll à=nós=\textbf{mā} ɓóm \\
 \textsc{cl2}=pot(\textsc{cl2})=\textbf{\textsc{gen.cl2}} woman(\textsc{cl2}) \\
\glt `The woman's pot.' \\
\z

\ea
 Associative \\
\gll à=rǐs kʼwāní \\
 \textsc{cl2}=many.\textsc{pl}(\textsc{cl2}) people(\textsc{cl1}) \\
\glt `Very many people' \\
\z

\ea
 Associative \\
\gll à=rǐs=\textbf{ā} kúnùʔ \\
 \textsc{cl2}=many.\textsc{pl}(\textsc{cl2})=\textbf{\textsc{ass.cl2}} owl(\textsc{cl2}) \\
\glt `Very many owls' \\
\z


\subsection{Prepositions, conjunctions, and complementizers}
\label{sec:Don:Prepositions_Conjunctions_Complementizers}

In addition to case marking, gender is also marked on prepositions,
conjunctions, and complementizers in Uduk through a simple phonological
alternation.  If a preposition ends in \emph{i}, this changes to \emph{a} before
Class 2 nouns, retaining the tone of the original vowel.  If a preposition ends
in a consonant or another vowel than \emph{i}, then \emph{a} attaches to the end
of the preposition.  As mentioned previously, if gender is marked on the
previous element, then Class 2 marker \emph{à} is not used.

These alternations are likely based a type of cliticization similar to case
markers, but slightly more grammaticalized.  Nonetheless, in occasional careful
speech with \emph{d̪àlì} `and, but' for instance, it is possible to hear
\emph{d̪àlì à} before Class 2 nouns instead of \emph{d̪àlà}.%
\footnote{Note that in the following examples, ø being marked as a clitic is just a notational
choice to facilitate understanding.}

\ea
\gll  ràkʰ tā-ø kúʃ mò \textbf{í} mīs \\
 cloud(\textsc{cl1}) \textsc{cop}:\textsc{pfv}-\textsc{3sg} white \textsc{mo} \textbf{\textsc{loc:cl1}} sky(\textsc{cl1}) \\
\glt `The clouds are white in the sky.' \\
\z

\ea
\gll  áhā wòl-á=ø yìɗé \textbf{á} kʰ\oMidLow{}\hspace*{-.4mm}s \\
 \textsc{1sg}(\textsc{cl1}) pour:\textsc{ipfv}-\textsc{1sg}=\textsc{cl1} water(\textsc{cl1}) \textbf{\textsc{loc:cl2}} cup(\textsc{cl2}) \\
\glt `I poured the water in the cup.' \\
\z

\ea
\gll  é gǎm-ø=ø tō yán \textbf{pʼéní} máná? \\
 \textsc{2sg}(\textsc{cl1}) find:\textsc{ipfv}-\textsc{3sg}=\textsc{cl1} thing(\textsc{cl1}) \textsc{dem.prox} \textbf{from:\textsc{cl1}} where(\textsc{cl1}) \\
\glt `Where did you get this thing from?' \\
\z

\ea
\gll  gǎm-kāʔ \textbf{pʼéná} Yúsìf \\
 find:\textsc{ipfv}-\textsc{erg.1sg} \textbf{from:\textsc{cl2}} Yousef(\textsc{cl2}) \\
\glt `I got (it) from Yousef.' \\
\z

\ea
\gll  áhā kʼwār-á \textbf{kā} gǎlām \\
 \textsc{1sg}(\textsc{cl1}) write:\textsc{ipfv}-\textsc{1sg} \textbf{with:\textsc{cl2}} pen(\textsc{cl2}) \\
\glt `I'm writing with a pen.' \\
\z

Predicative possession constructions also index gender of the possessed noun on
a preposition-like marker.  Predicative possessive constructions are formed with
the copula \emph{tā} along with the particle \emph{gì}, which becomes
\emph{gà} before Class II nouns. \nocite{James1979}

\ea
\gll  wàt̪í tā \textbf{gì} mì \\
 man(\textsc{cl1}) \textsc{cop}:\textsc{pfv} \textbf{\textsc{pp.cl1}} goat(\textsc{cl1}) \\
\glt `The man has a goat.' \\
\z

\ea
\gll  áhā tā-ná \textbf{gà} kʼá \\
 \textsc{1sg}(\textsc{cl1}) \textsc{cop}:\textsc{pfv}-\textsc{1sg} \textbf{\textsc{pp.cl2}} dog(\textsc{cl2}) \\
\glt `I have a dog.' \\
\z

Conjunctions and complementizers are preposition-like words used to connect
clauses or phrases. Similar to prepositions, the gender of the immediately
following word is marked on the conjunction or complementizer by an alternation
of \emph{i} to \emph{a} for words ending in \emph{i}, or by adding \emph{a} to
the end of words which end in consonants or vowels other than \emph{i}.

The most frequent of these is \emph{kí}, or \emph{ká} for Class 2 nouns. It
is a general complementizer which occurs with many different types of complement
phrases and clauses, as well as subordinate clauses.

\ea
\gll  áhā tʰōʃ-á \textbf{ká} ʃōkʼ mì-ɗ=ì hét̪ʼ {kā t̪ʼámō} \\
 \textsc{1sg}(\textsc{cl1}) think:\textsc{ipfv}-\textsc{1sg} \textbf{\textsc{comp.cl2}} rain(\textsc{cl2}) do.\textsc{aux}:\textsc{ipfv}-\textsc{3sg}=\textsc{lnk} rain\textsubscript{verb} tomorrow \\
\glt `I hope it rains tomorrow.' \\
\z

\ea
\gll  áhā tʰōʃ-á \textbf{kí} wàt̪í mǐ-ɗ=ì tʼā kí pʰúɗ mò ʃwànéʔ \\
 \textsc{1sg}(\textsc{cl1}) think:\textsc{ipfv}-\textsc{1sg} \textbf{\textsc{comp:cl1}} man(\textsc{cl1}) do.\textsc{aux}:\textsc{ipfv:ad2}-\textsc{3sg}=\textsc{lnk} \textsc{cf.aux} \textsc{comp} arrive \textsc{mo} today \\
\glt `I thought that the man would have arrived today.' \\
\z

With some adverbial phrase constructions, \emph{kī} and \emph{kā} with mid tones are used instead of \emph{kí} and \emph{ká} with high tones.\nocite{d:BeamCridland:Uduk}

\ea
\citep{d:BeamCridland:Uduk}\\
\gll  jàmàs bǔnī kʼó-n \textbf{kā} rìs \\
 kind(\textsc{cl1}) \textsc{poss.3pl} exist.\textsc{pl}:\textsc{pfv}-\textsc{3pl} \textbf{with:\textsc{cl2}} many(\textsc{cl2}) \\
\glt `There are many kinds of them.' \\
\z

\ea
\gll  únī dǒʃ-ón \textbf{kī} mís \\
 \textsc{3pl} stand:\textsc{ipfv}-\textsc{3pl} \textbf{with:\textsc{cl1}} sky(\textsc{cl1}) \\
\glt `They stood up.' \\
\z

There are three additional subordinating conjunctions: \emph{wàkʰkí} for
conditional clauses, \emph{gòm} for reason and adversative clauses, and
\emph{mèɗ} for temporal clauses. All of these alternate according to the
gender of the noun which follows in the manner described above.

\ea
\gll  \textbf{wàkʰkí} wàt̪í kʼóʃ-óɗ=ā shētʰ, kʼúpʰ tō mí-nù mí=ì kʰál bwày cōm=á? \\
 \textbf{if:\textsc{cl1}} person(\textsc{cl1}) kill:\textsc{pfv}-\textsc{3sg}=\textsc{acc.cl2} antelope(\textsc{cl2}), head(\textsc{cl1}) thing(\textsc{cl1}) do.\textsc{aux}:\textsc{pfv}-\textsc{imprs} do.\textsc{aux}=\textsc{lnk} carry to:\textsc{cl1} his.father(\textsc{cl1})=\textsc{q} \\
\glt `If a person kills an antelope, is the head carried to the father's home?' \\
\z

\ea
\label{ex:Don:reach-year}
\gll \textbf{wàkʰká} cʼí pʰúɗ-úɗ mò yìl {kʼúmèɗ pé kwārā} áw̄ {kʼúmèɗ ì pé súʔ} áɗī kí tʰél mí pʼén=ì màʃ mò\\
 \textbf{if:\textsc{cl2}} child(\textsc{cl2}) reach:\textsc{ipfv}-\textsc{3sg} \textsc{mo} year(\textsc{cl1}) thirteen or twelve \textsc{3sg}(\textsc{cl1}) \textsc{narr} begin do.\textsc{part} behind.\textsc{part}=\textsc{lnk} marry \textsc{mo}\\
\glt `If the child reaches the year thirteen or twelve then he can start to get married.' \\
\z

The only native coordinating conjunction is \emph{d̪àlì} (Class 2
\emph{d̪àlà}) `and; but', and is very frequent.%
\footnote{Two other conjunctions borrowed from Arabic also exist: \emph{wàlà}
and \emph{áw̄}, both meaning `or (used to rephrase something)'.  Neither term
alternates according to the gender of the noun which follows.} %
It may coordinate clauses, noun phrases, and nouns.

\ea
\gll  \textbf{d̪àlì} tōntʰéʔ yǐs\aMidLow{} dì-ɗ yǐsā=yà \\
 \textbf{and:\textsc{cl1}} food(\textsc{cl1}) \textsc{neg} exist.\textsc{sg}:\textsc{pfv}-\textsc{3sg} \textsc{neg}=\textsc{neg} \\
\glt `And there was no food.' \\
\z

\ea
(\citealt{James1979}, The Birapinya Tree)\\
\gll  \textbf{d̪àlà} ɓóm ɲǎŋ-ø=ø gùɓ ʃēmēn bwày \\
 \textbf{and:\textsc{cl2}} woman(\textsc{cl2}) build:\textsc{ipfv}-\textsc{3sg}=\textbf{\textsc{cl1}} house(\textsc{cl1}) alongside:\textsc{cl1} road(\textsc{cl1}) \\
\glt `and a woman had built her house alongside the road.' \\
\z


\subsection{Prenominal modifiers}
\label{sec:Don:Prenominal_modifiers}

Out of all the prenominal modifiers, two of them index the gender of the noun
they modify, namely the diminutive \emph{ārí} and its irregular plural form
\emph{ūʃí}.  Both the singular as well as the plural diminutive are lexically
nouns themselves, with inherent gender (Class 1).  However, they alternate their
final vowel according to the gender of the following noun: \emph{í} before Class
1, and \emph{á} before Class 2.

\ea
\gll  áhā mìʃ-á=ø \textbf{ārí} mì \\
 \textsc{1sg}(\textsc{cl1}) see:\textsc{ipfv}-\textsc{1sg}=\textbf{\textsc{cl1}} \textbf{\textsc{dim:cl1}(\textsc{cl1})} goat(\textsc{cl1}) \\
\glt `I saw the little goat.' \\
\z

\ea
\gll  áhā mìʃ-á=ø \textbf{ārá} ɲǎw \\
 \textsc{1sg} see:\textsc{ipfv}-\textsc{1sg}=\textbf{\textsc{cl1}} \textbf{\textsc{dim:cl2}(\textsc{cl1})} cat(\textsc{cl2}) \\
\glt `I saw the little cat.' \\
\z

There is one special case in regards to prenominal modifiers that should also be mentioned, one of the only instances of non-adjacent indexation of gender. When
prenominal modifiers modify a postverbal A argument, the verb does not agree
with the inherent gender of the modifier, but rather with the noun that the
prenominal modifier is modifying.

\ea
 Class I Noun \\
\gll à=ɓóm mìʃ=\textbf{à} wàt̪íʔ \\
 \textsc{cl2}=woman(\textsc{cl2}) see:\textsc{ipfv}=\textbf{\textsc{erg.cl1}} man(\textsc{cl1}) \\
\glt `The man sees the \uline{woman}.' \\
\z

\ea
 Class I Modifier, Class I Noun \\
\gll à=ɓóm mìʃ=\textbf{à} d̪àn wàt̪íʔ \\
 \textsc{cl2}=woman(\textsc{cl2}) see:\textsc{ipfv}=\textbf{\textsc{erg.cl1}} big(\textsc{cl1}) man(\textsc{cl1}) \\
\glt `The big man sees the \uline{woman}.' \\
\z

\ea
 Class II Noun \\
\gll wàt̪íʔ mìʃ=\textbf{mà} ɓóm \\
 man(\textsc{cl1}) see:\textsc{ipfv}=\textbf{\textsc{erg.cl2}} woman(\textsc{cl2}) \\
\glt `The woman sees the \uline{man}.' \\
\z

\ea
 Class I Modifier, Class II Noun \\
\gll wàt̪íʔ mìʃ=\textbf{mà} d̪àn=\textbf{à} ɓóm \\
 man(\textsc{cl1}) see:\textsc{ipfv}=\textbf{\textsc{erg.cl2}} big(\textsc{cl1})=\textbf{\textsc{ass.cl2}} woman(\textsc{cl2}) \\
\glt `The big woman sees the \uline{man}.' \\
\z

Constructions of this type have only appeared in elicited circumstances,
however, and speakers appeared to be somewhat reluctant to use them.  Not all
Uduk speakers would necessary find these grammatical; many would find them odd,
at the very least, and would avoid using postverbal A arguments with prenominal
modifiers.

\subsection{Verbs}
\label{sec:Don:Verbs}

Finite verbs are the last target for gender indexation presented here; verbs
indicate the gender of O arguments through a rather unusual fashion.

In constructions in which the O argument is Class 2 (e.g.\ marked with the
Accusative), the A argument is cross-referenced in the same way that S would be
in monovalent clauses. Verbs with a 3SG subject are marked with \emph{-(V)ɗ},
and verbs with a 2SG, 2PL, or 3PL subject are marked with \emph{-(V)n} on the
verb.

\ea
\label{ex:Don:cut-pelt}
Class 2 O, 3SG person subject \\*
\gll wàt̪í cʼít̪ʼ-íɗ=\textbf{ā} yíɗ \\
 man(\textsc{cl1}) cut:\textsc{ipfv}-\textsc{3sg}=\textbf{\textsc{acc.cl2}} skin(\textsc{cl2}) \\
\glt `The man is cutting the pelt.' \\
\z

\ea
\label{ex:Don:find-baboons}
Class 2 O, 3PL person subject \\
\gll únī gǎm-án=\textbf{ā} dàwà kā rìs \\
 \textsc{3pl}(\textsc{cl1}) find:\textsc{ipfv}-\textsc{3pl}=\textbf{\textsc{acc.cl2}} baboon(\textsc{cl2}) with:\textsc{cl2} many(\textsc{cl2}) \\
\glt `They found many baboons.' \\
\z

\ea
\label{ex:Don:find-child}
Class 2 O, 2SG person subject \\
\gll é gǎm-án=\textbf{ā} cʼí \\
 \textsc{2sg}(\textsc{cl1}) find:\textsc{ipfv}-\textsc{2sg}=\textbf{\textsc{acc.cl2}} child(\textsc{cl2}) \\
\glt `You have found the child.' \\
\z

\ea
 Class 2 O, 1SG person subject \\
\gll áhā pʰī-ná=\textbf{ā} sū \\
 \textsc{1sg}(\textsc{cl1}) drink:\textsc{ipfv}-\textsc{1sg}=\textbf{\textsc{acc.cl2}} beer(\textsc{cl2}) \\
\glt `I am drinking the beer.' \\
\z

Class 1 O arguments not only do not take overt Accusative marking, but they also
trigger a reduction of verbal morphology.  Subject cross-referencing markers on
the verb for second and third person A arguments are suppressed,%
\footnote{Under normal circumstances, it is not possible for any other element
to intervene between the verb and the noun that follows.  There is one
instance in my database pointed out to me by a reviewer (example
\ref{ex:Don:reach-year}), however, in which the aspectual marker \emph{mò} does
come in between a verb and a Class 1 noun.  In this instance, cross-referencing
of A on the verb is actually realized, suggesting that there may be additional
factors involved in the suppression of the second/third person suffix.  More
research is needed to determine if this is indeed the case, and if so, what
those might be.  This may simply be an intransitive clause, with `year'
functioning adverbially.} %
and cross-referencing on the verb only appears with first person subjects.


\ea
\label{ex:Don:cut-cloth}
Class 1 O, 3SG person subject \\
\gll áɗī cʼít̪ʼ-ø=\textbf{ø} bùɲjè \\
 \textsc{3sg}(\textsc{cl1}) cut:\textsc{ipfv}-\textsc{3sg}=\textbf{\textsc{acc.cl1}} cloth(\textsc{cl1}) \\
\glt `S/he's cutting the cloth.' \\
\z

\ea
\label{ex:Don:pick-bowl}
Class 1 O, 3PL person subject \\
\gll únī ɗékʼ-ø=\textbf{ø} kʼwā \\
 \textsc{3sg}(\textsc{cl1}) pick\_up:\textsc{ipfv}-\textsc{3sg}=\textbf{\textsc{acc.cl1}} bowl(\textsc{cl1}) \\
\glt `They picked up the bowl.' \\
\z

\ea
\label{ex:Don:found-thing}
Class 1 O, 2SG person subject \\
\gll é gǎm-ø=\textbf{ø} tō yán \\
 \textsc{2sg}(\textsc{cl1}) find:\textsc{ipfv}-\textsc{3sg}=\textbf{\textsc{acc.cl1}} thing(\textsc{cl1}) \textsc{dem.prox} \\
\glt `You found this thing.' \\
\z

\ea
 Class 1 O, 1SG person subject \\
\gll áhā pʰī-ná=\textbf{ø} yìɗé \\
 \textsc{1sg}(\textsc{cl1}) drink:\textsc{ipfv}-\textsc{1sg}=\textbf{\textsc{acc.cl1}} water(\textsc{cl1}) \\
\glt `I am drinking the water.' \\
\z

Examples (\ref{ex:Don:cut-cloth}), (\ref{ex:Don:pick-bowl}), and
(\ref{ex:Don:found-thing}) are parallel to (\ref{ex:Don:cut-pelt}),
(\ref{ex:Don:find-baboons}), and (\ref{ex:Don:find-child}) in structure, but
with the subject cross-referencing markers on the verb suppressed.

First person subjects on the other hand do not change their cross-reference
marking, irrespective of the gender of O.  The only indication of the gender of
O in these constructions is the ACC marker.

\ea
 Class 2 O, 1SG person subject \\
\gll áā mìʃ-á=\textbf{ā} wùlúʔ mò \\
 \textsc{1sg} see:\textsc{ipfv}-\textsc{1sg}=\textbf{\textsc{acc.cl2}} tawny.eagle(\textsc{cl2}) \textsc{mo} \\
\glt `I saw a tawny eagle.' \\
\z

\ea
\label{ex:Don:know-place}
Class 1 O, 1SG person subject; \cite{d:BeamCridland:Uduk} \\
\gll áhā mìʃ-á=\textbf{ø} mò gì dǐ-n=ā áɗī \\
 \textsc{1sg}(\textsc{cl1}) see:\textsc{ipfv}-\textsc{1sg}=\textbf{\textsc{acc.cl1}} place(\textsc{cl1}) \textsc{gen.rel} exist.\textsc{sg}:\textsc{ipfv}-\textsc{nas}=\textsc{erg.cl1} \textsc{3sg}(\textsc{cl1}) \\
\glt `I know the place where he is.'  \\
\z


The phenomenon described above does not apply to Narrative constructions, where
arguments are never cross-referenced on the verb.  Narrative constructions use
non-finite forms of verbs, and the only difference between Narrative
constructions with Class 1 objects and Narrative constructions with Class 2
objects is the Accusative case marker.


\ea
 Class 1 O, Narrative construction \\
\gll à=cí kí kʼósh=\textbf{ø} wàt̪í mò \\
 \textsc{cl2}=creature(\textsc{cl2}) \textsc{narr} hit\textsubscript{\textsc{nf}}=\textbf{\textsc{acc.cl1}} person(\textsc{cl1}) \textsc{mo} \\
\glt `He attacks the man.' \\
\z


\ea
 Class 2 O, Narrative construction \\
\gll áʼdī kí bùt̪=\textbf{à} cʼí d̪àlì kʼósh=\textbf{ā} cʼí mò \\
 \textsc{3sg}(\textsc{cl1}) \textsc{narr} catch\textsubscript{\textsc{nf}}=\textbf{\textsc{acc.cl2}} child(\textsc{cl2}) and hit\textsubscript{\textsc{nf}}=\textbf{\textsc{acc.cl2}} child(\textsc{cl2}) \textsc{mo} \\
\glt `She catches the child and beats the child.' \\
\z

Note that personal pronouns have inherent Class 1 gender,%
\footnote{Described more fully in \sectref{sec:Assignment} below.} %
and the gender of a pronoun does not reflect the gender of the noun it denotes.

\ea
\gll  à=kʰúrā lǒɓ=\textbf{mā} cʼí \\
 \textsc{cl2}=ball(\textsc{cl2}) play:\textsc{ipfv}=\textbf{\textsc{erg.cl2}} child(\textsc{cl2}) \\
\glt `The child is playing \uline{(foot)ball}.' \\
\z

\ea
\gll  à=kʰúrā lǒɓ=\textbf{ā} áɗī \\
 \textsc{cl2}=ball(\textsc{cl2}) play:\textsc{ipfv}=\textbf{\textsc{erg.cl1}} \textsc{3sg}(\textsc{cl1}) \\
\glt `S/he is playing \uline{(foot)ball}.' \\
\z


Pronominal objects also trigger indexation patterns in which second and third person
cross-referencing of A is suppressed.

\ea
 Class 2 O, 3SG person subject \\
\gll wàt̪í kʼōʃ-óɗ=\textbf{ā} Rǎbì \\
 man(\textsc{cl1}) hit:\textsc{ipfv}-\textsc{3sg}=\textbf{\textsc{acc.cl2}} Rabi(\textsc{cl2}) \\
\glt `The man hit Rabi.' \\
\z

%contour tone in line 811
\ea
 Class 1 O, 3SG person subject \\
\gll wàt̪í kʼ\oMidHigh{}\hspace*{-0.4mm}ʃ-ø=\textbf{ø} áɗī \\
 man(\textsc{cl1}) hit:\textsc{ipfv}-\textsc{3sg}=\textbf{\textsc{acc.cl1}} \textsc{3sg}(\textsc{cl1}) \\
\glt `The man hit him/her/it.' \\
\z


\section{Gender assignment}
\label{sec:Assignment}

Gender assignment in Uduk is largely, but not exclusively, arbitrary, with only
limited connections to semantic categores such as biological sex, size, shape,
and animacy. There are no distinctions based on sex, human vs.\ non-human, or
animate vs.\ inanimate, and neither sex nor animacy is distinguished in the
pronominal system for any person.

Nouns generally considered among the highest in the animacy scale, such as human
kinship terms, do not show transparent assignment.

A list of human nouns and their gender may be found in Table
\ref{tab:human_nouns}, with little or no predictability beyond the fact that
most suppletive possessive kinship terms appear to fall into Class 1.

\begin{table}[p]
\centering
\caption{Class 1 and Class 2 human nouns}
\small
\begin{tabular}{>{\itshape}ll}
\lsptoprule
\multicolumn{2}{c}{Class 1} \\
\hline
wàt̪íʔ & man \\
yàʔ & son \\
ɓwāhām & female sibling or parallel cousin \\
ɓwāʔ & daughter \\
āʃ & wife \\
jìl & sisters-in-law, recip. \\
kūm & his, her mother \\
kwān & your mother \\
cím & your father \\
cōm & his, her father \\
sóɓ & his, her father's sister \\
nà(ḿ) & niece, nephew (sister's children) \\
símín & father's sister \\
yàʃím & brother's wife; husband's brother or sister \\
kʼwáskām & cross-cousin \\
kʼwáskīn & your cross-cousin \\
\hline
\multicolumn{2}{l}{all personal pronouns} \\
\multicolumn{2}{l}{all plural derived agentive nouns} \\
\hline
\multicolumn{2}{c}{Class 2} \\
\hline
à=ɓóm & woman, wife \\
à=kām & male sibling or parallel cousin \\
à=bàpá & father \\
à=táɗā & mother \\
à=màmá & my mother, also vocative \\
à=kāt̪ʰ & husband \\
à=cʼí & child (general) \\
à=mǎmà & father's sister \\
à=tātʰá & mother's brother \\
à=ʃwákām & mother's brother \\
à=nàrú & mother's brother \\
à=ʔíyā & father's brother; brother's children \\
à=màrè & mother and father-in-law, for man \\
à=màr & mother and father-in-law, for woman \\
à=màséʔ & sister's husband \\
à=mʷí & sister's children (for men) \\
à=dìtʰíʔ & elderly woman, esp.\ father's sister \\
à=mòɲérù & second cousin, more distant relationship \\
à=ɲèrgòn & cousin (telling to third person) \\
\hline
\multicolumn{2}{l}{all personal names, male and female} \\
\multicolumn{2}{l}{all singular derived agentive nouns} \\
\lspbottomrule
\end{tabular}
\label{tab:human_nouns}
\end{table}

\citet[101]{Dahl2000a} postulates the following:

\begin{enumerate}
\item In any gender system, there is a general semantically-based principle for assigning gender to animate nouns and noun phrases.

\item The domain of the principle referred to in (1) may be cut off at different points of the animacy hierarchy: between humans and animals, between higher and lower animals, or between animals and inanimates.
\end{enumerate}

That is, by using a hierarchy such as the one found in Table \ref{tab:animacy},
one can make predictions on what types of gender systems may occur, and where
semantically-based principles apply.  Dahl suggests that cross-linguistic
cut-off points vary, but always take place below human.

\begin{figure}[htb]
\setlength{\tabcolsep}{.15em}
\caption{Animacy hierarchy}
\label{tab:animacy}
\begin{tabular}{|c l l l l|} \hline
1st person & > 2nd person & > 3rd person & > proper names & > kin \\
 & > other humans &  > animate nouns & > inanimate nouns & \\
\hline
\end{tabular}
\end{figure}


Semantic assignment is not predictable for human apellatives in Uduk; however,
there \emph{are} semantic areas in which predictability does occur: namely
personal (and demonstrative) pronouns as well as proper names, both categories
above human in the animacy hierarchy.

All personal pronouns show gender assignment in the same way that nouns do, and
could be considered a lexical subtype of nouns.  Demonstratives and personal
pronouns are all assigned to the nominal Class 1 gender; they show no connection
to the gender of a noun in anaphoric contexts, and are invariably Class 1.  This
is partially comparable to Jarawara (Arawan), in which ``all pronouns (whatever
the sex of their referent) engender feminine agreement on verbal suffixes''
\citep[488]{Dixon2000}.  Proper names on the other hand are assigned
to Class 2.  This generalization holds only for personal names; place names can
vary.  Uduk gender predictability thus appears to apply only to levels higher
than human appellatives in the animacy hierarchy.

Below this cut-off point there are limited trends in semantic assignment, but
the semantic groups that can be formed all have exceptions.  Nouns denoting
plural entities, \emph{kʼwāní} `people', \emph{ūpʰ} `women', and
\emph{ūcʰí} `children', are Class 1.  Furthermore, a limited subset of nouns
(primarily proper names and some kinship terms) in Uduk may appear with the
Associative Plural (AP) prefix \emph{ī-} to denote a person and additional
people associated with that person; nouns marked in this way are also Class 1.
This includes plurals which would otherwise be assigned to Class 2, such as
proper names.%
\footnote{Note that most nouns in Uduk are not normally morphologically
marked for number; the Associative Plural is one of the only means of marking
number directly on a noun, and even this is only possible to use with a limited
set of nouns.}

Most relational nouns, nouns which are primarily used to indicate more detailed
types of spatial or temporal relationships, are also Class 1. This includes
nouns like \emph{ʃēmén} `alongside', \emph{pʼémèn} `end, bottom (of)',
\emph{bwàmán} `inside, between', \emph{bwàmbòr} `front (of)'; a few, such as
\emph{à=pʰóʔ} `on top of' and \emph{à=píjè}, `outside' are Class 2. Lastly,
body parts are also more commonly found in Class 1 than Class 2.

Formal assignment in terms of word formation rules also creates limited
situations in which gender assignment may be predicted. Nominalizations of
Stative verbs, marked with the suffix \emph{-gàʔ}, are invariably assigned to
Class 2. Agentive nouns formed with the derivational morpheme \emph{màn} are
also assigned to Class 2. Nouns derived from verbs which use zero derivation,
however, are all assigned to the Class 1 gender.

Uduk nouns tend to be fairly rigid in their assignment of gender, and few
lexemes seem to have the possibility of occurring in either class. In these
instances, there is no change in meaning. This includes intraspeaker variation
as well as free variation within the speech of the same speakers.

There are a few instances in which homophonous nouns are assigned to different
classes, e.g.\ \emph{jè}, `elephant', and \emph{à=jè}, `mud; type of fish',
but these are purely lexical distinctions, and remain rigid in assignment.

There is a markedness relationship between the two classes. In many respects,
Class 1 could be considered the unmarked, default class, particularly for less
nouny nouns, such as pronouns. In addition to the lack of overt morphology in
many instances, there are other signs that Class 1 is seen as the default.
Conjunctions which occur before word classes other than nouns, for instance, use
the same form as before Class 1 nouns.
%contour tones in line 976, 977
However, in other respects, Class 2 could also be considered a default.  Class 2
is the default for nouns and adjective-like concepts, and a large number
(although not all) of borrowed words appear to be placed into Class 2, e.g.\
\emph{à=bǎsàl} `onion', \emph{à=bìʃk\iLowMid{}\hspace*{-0.6mm}r} `towel',
\emph{à=màsábà} `distance', \emph{á=ʃáb\aLowMid{}\hspace*{-.4mm}gà},
`network'.

\section{Complexity}
\label{sec:Complexity}

Uduk shows itself to have an atypical gender system, and it is worth
investigating its complexity in more detail, and how it might compare.
\citet[183]{DiGarbo2014} uses six features to determine the
complexity of a gender system: Number of gender values, Nature of assignment
rules, Number of targets, Cumulative exponence of gender and number,
Manipulation of gender assignment triggered by number/countability, and
Manipulation of gender assignment triggered by size.

By these features as well as some others, Uduk has a relatively
simple system. There are only two genders, to which nouns are generally rigidly
assigned. No manipulation is possible, and aside from the Associative Plural
marker, there are no instances in which number and gender are marked
cumulatively. There are three targets: case marking particles, verbs, and
adpositions/conjunctions/complementizers (which all form part of a single
category), and a marginal fourth in the form of the diminutive (not included
here as it does not constitute a word class; see \sectref{sec:Don:Prenominal_modifiers}). Assignment parameters feature
higher complexity, however, as assignment is partly semantic, partly formal, and
partly opaque.

There were two additional criteria mentioned in \sectref{sec:Introduction},
arbitrariness in gender assignment and adjacency, which play an interesting role
in complexity, although at the moment it is difficult to see precisely how to
reconcile them in terms of complexity metrics.

In nearly all instances in which gender is indexed on a target in Uduk, the
gender-marked target and controller are immediately adjacent, with the target in
the immediate position before the controller.  This adds slightly to the
descriptive complexity, as it requires an extra rule or constraint specifying
this in the description.

Arbitrariness in gender assignment is even more difficult to reconcile, but an
arbitrary system is likely also more complex.  In principle, assignment would
reach maximal complexity if each individual noun required a separate descriptive
rule.

Both arbitrariness of assignment as well as adjacency require further research
in general.  Whether we exclude or include these as factors, however, it would
appear that Uduk does have a relatively simple gender system, albeit atypical.

\section{Discussion}
\label{sec:Discussion}

The Uduk gender system turns out to have a number of intriguing aspects.
First, the system makes heavy use of zero marking and in one instance,
suppression of subject agreement morphemes to indicate the gender of an object.

Second, almost all targets of indexation are adjacent to the controller. This
is not commonly remarked upon cross-linguistically,%
\footnote{One important exception to this is Bernhard Wälchli's work on Nalca
\citep{Waelchli2018}.  Wälchli was also the one who pointed out
adjacency as a relevant factor in Uduk to me, and I likely would not have
noticed or remarked upon this without his input.  Additionally, ǃXóõ also
appears to index gender only on adjacent targets; for further details, see
\citet{Gueldemann2006}.} %
and by making note of it here, it may encourage other linguists to explore
adjacency as a factor at play in gender marking systems.

Third, personal and demonstrative pronouns control gender in the same way that
nouns do.  And finally, gender is not connected to biological sex or other
familiar semantic categories.

As mentioned previously, the last two characteristics are connected in Uduk.
Semantic predictability in Uduk occurs at higher levels of animacy than simply
human.  It parallels some Austronesian languages such as Tagalog and Fijian for
instance, which Hockett described as having gender, although later linguists
have not.

\begin{quotation}
In Fijian, /mata/ `day' is preceded by /na/ when it is the subject of a clause,
but /viti/ 'Fiji' is preceded instead by /ko/. /na/ and /ko/ are two distinct
particles, not different inflected forms of a single stem. Yet the choice of
/na/ or /ko/ establishes a twofold classification of all Fijian nouns and noun
phrases: names of specific people and places belong to the /ko/ class, common
nouns to the /na/ class. \citep[230]{Hockett1958}
\end{quotation}

\noindent Even more interestingly,
``...independent pronouns [in Fijian] function in many ways like proper nouns, and are
frequently marked by the same marker (\textit{ko} or \textit{o})'' \citep[201]{Geraghty-1983-Fijian}.

\begin{table}[htb]
\centering
\caption{Noun phrase markers and pronouns in Tagalog \citep[358]{Himmelmann_Tagalog2005}}
\label{tab:Don:Tagalog}
\begin{tabular}{l*{3}{>{\itshape}l}}
\lsptoprule
 & \normalfont SPEC & \normalfont POSS/GEN & \normalfont LOC/DAT \\
Common nouns & ang & ng & sa \\
Personal names & si & ni & kay \\
\midrule
1SG & akó & ko & akin \\
2SG & ikáw, ka & mo & iyo, iyó \\
3SG & siyá & iyá & kaniyá \\
1DU.IN & kitá, katá & nitá & kanitá \\
1PL.IN & tayo & natin & atin \\
1PL.EX & kamí & namin & amin \\
2PL & kayó & ninyó & inyó \\
3PL & silá & nilá & kanilá \\
\midrule
PROX & itó & nitó & dito, rito \\
MED & iyán & niyán & diyán, riyán \\
DIST & iyón & niyón, noón & doón, roón \\
\lspbottomrule
\end{tabular}
\end{table}

A comparable system is found in Tagalog (\tabref{tab:Don:Tagalog}), which could also be viewed as having a
common vs.\ proper gender system.  Tagalog additionally has distinct forms for
demonstratives and each pronoun, suggesting that these are internally viewed as
a third category, neither common nor proper (and different from Fijian in this
respect).

In both cases, Tagalog and Fijian have a higher cut-off point in animacy than
human nouns, requiring a more fine-grained approach to the animacy hierarchy.
This cut-off point appears to show some parallels to Uduk.  Where Fijian for
instance differs from Uduk, however, is that in Uduk, proper names and personal
pronouns do not occur in the same gender, and thus a proper-common gender
differentiation would not be suitable as an analysis.  Uduk would instead show
two genders, one consisting of personal and demonstrative pronouns and other
nouns, and the other consisting of proper names and other nouns.

Languages like Tagalog, Fijian, and Uduk give evidence suggesting that
predictability may occur at points higher in the animacy hierarchy than
previously acknowledged, although Uduk shows itself to be more complex than
Tagalog or Fijian, as the gender of its nouns are generally much less
predictable. By including Uduk as a typological point of reference, a
reconsideration of possible cut-off points in the animacy hierarchy may be in
order.

\section*{Abbreviations}
\begin{tabularx}{\textwidth}{lXlX}
  \textsc{ad1} 	&	 Aspect-Directional 1 	&	\textsc{dem} 	&	 Demonstrative 	\\
  \textsc{ad2} 	&	 Aspect-Directional 2 	&	\textsc{dim} 	&	 Diminutive 	\\
  \textsc{ass} 	&	 Associative 	&	\textsc{lnk} 	&	 Linker 	\\
  \textsc{aux} 	&	 Auxiliary 	&	\textsc{mo} 	&	 Aspect-mood particle 	\\
  \textsc{cf} 	&	 Counterfactual 	&	\textsc{narr} 	&	 Narrative 	\\
  \textsc{cl1} 	&	 Class 1 Gender 	&	\textsc{nas} 	&	 Nasal 	\\
  \textsc{cl2} 	&	 Class 2 Gender 	&	\textsc{nf} 	&	 Non-finite 	\\
  \textsc{comp} 	&	 Complementizer 	&	\textsc{part} 	&	 Partargument 	\\
  \textsc{cop} 	&	 Copula 	&	\textsc{pp} 	&	 Predicative Possession 	\\
\end{tabularx}

\section*{Acknowledgements}

I would like to thank the Kone foundation, who financially supported this research, and the UH-SU cooperation project, who financially supported the workshop on Grammatical Gender and Linguistic Complexity. I would also like to thank Bernhard Wälchli and Francesca Di Garbo for all their time and effort as both reviewers as well as editors; they contributed a great deal in helping me develop the ideas presented here. I also thank Bruno Olsson for his help and technical assistance in editing and layout formating, and Johanna Nichols, Manuel Otero, the anonymous reviewers, and the participants of the Grammatical Gender and Linguistic Complexity workshop for their feedback and comments. Last but not least, I would also like to thank all of my Uduk consultants, who devoted a lot of time and effort helping me understand their language.

Any remaining errors are of course the author's own responsibility.


\printbibliography[heading=subbibliography]


\label{lastpage:Killian}
\end{document}
